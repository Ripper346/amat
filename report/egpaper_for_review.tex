\documentclass[10pt,twocolumn,letterpaper]{article}

\usepackage{iccv}
\usepackage{times}
\usepackage{epsfig}
\usepackage{graphicx}
\usepackage{amsmath}
\usepackage{amssymb}

% Include other packages here, before hyperref.

% If you comment hyperref and then uncomment it, you should delete
% egpaper.aux before re-running latex.  (Or just hit 'q' on the first latex
% run, let it finish, and you should be clear).
\usepackage[pagebackref=true,breaklinks=true,letterpaper=true,colorlinks,bookmarks=false]{hyperref}

% \iccvfinalcopy % *** Uncomment this line for the final submission

\def\iccvPaperID{****} % *** Enter the ICCV Paper ID here
\def\httilde{\mbox{\tt\raisebox{-.5ex}{\symbol{126}}}}

% Pages are numbered in submission mode, and unnumbered in camera-ready
\ificcvfinal\pagestyle{empty}\fi
\begin{document}

%%%%%%%%% TITLE
\title{\LaTeX\ Author Guidelines for ICCV Proceedings}

\author{First Author\\
Institution1\\
Institution1 address\\
{\tt\small firstauthor@i1.org}
% For a paper whose authors are all at the same institution,
% omit the following lines up until the closing ``}''.
% Additional authors and addresses can be added with ``\and'',
% just like the second author.
% To save space, use either the email address or home page, not both
\and
Second Author\\
Institution2\\
First line of institution2 address\\
{\tt\small secondauthor@i2.org}
}

\maketitle
%\thispagestyle{empty}


%%%%%%%%% ABSTRACT
\begin{abstract}
\end{abstract}

%%%%%%%%% BODY TEXT %%%%%%%%%%%%%%%%%%%%%%%%%%%%%%%%%%
% ==========================================================
\section{Introduction}\label{sec:introduction}
% ==========================================================
Add MAT-, perceptual grouping-, and symmetry-related works here.
Also, application in which MAT has been used (shape matching, shape retrieval etc)

% ==========================================================
\section{Related Work}\label{sec:related}
% ==========================================================
\subsubsection*{MAT for binary shapes}
\begin{itemize}
\item Blum's Medial axis transform
\item Other works on extracting skeletons for 2D and 3D shapes (shock graphs,bone graphs,scale axis transform,check out other graphics works)
\end{itemize}
\subsubsection*{MAT for natural images}
\begin{itemize}
\item Levinstein's work
\item MIL-based symmetry detection
\item Random-forest variant by Alloimonos' student
\item CNN-variant 
\end{itemize}

% ==========================================================
\section{Background}\label{sec:background}
% ==========================================================
\subsubsection*{Medial Axis Transform}
\subsubsection*{Geometric Set Cover}

% ==========================================================
\section{Method}\label{sec:method}
% ==========================================================
\subsection{Appearance-MAT definition}\label{sec:amat}
\subsection{AMAT as a Geometric Set Cover Problem}
\subsection{Extending AMAT to Textured Images}
\begin{itemize}
\item $L_0$ Smoothing
\item GANs (if that works)
\end{itemize}

% ==========================================================
\section{Experiments}\label{sec:experiments}
% ==========================================================
\subsection{Image Reconstruction}
Experiments on image reconstruction using standard metrics (MSE, PSNR, and metrics using CNN feature similarity.

\subsubsection*{Abstract Scenes}
\subsubsection*{BSDS500}
This can only be done if we can make the GAN work and generate texture.
\subsubsection*{MSCOCO/Imagenet}
This can only be done if we can make the GAN work and generate texture.

\subsection{Medial Axis extraction (BSDS500)}
Recover all medial axes of the BSDS500. Compare with other skeletonization algorithms, such
as the CNN based method, and my previous MIL-based method

\subsection{Boundary detection (BSDS500)}
Medial axes to boundaries. 
MIL: compute an approximate scale based on the scale with maximum probability and try to recover boundaries.
CNN: same as above, this was already done in that paper for estimating segmentation masks using scale information.
We can compare the boundary recovery results as well.
EDGE DETECTORS: Standard edge detection algorithms (Dollar, HED, Canny, gPB etc: compare with algorithms specifically
tailored for this task. WE ARE NOT EXPECTING SOTA PERFORMANCE.

% ==========================================================
\section{Discussion}\label{sec:discussion}
% ==========================================================

{\small
\bibliographystyle{ieee}
\bibliography{egbib}
}

\end{document}
