\documentclass[10pt,twocolumn,letterpaper]{article}
% PACKAGES ===================================================================================
\usepackage{iccv}
\usepackage{times}
\usepackage{epsfig}
\usepackage{graphicx}
\usepackage{amsmath,amssymb}
\usepackage{algorithm,algpseudocode}
\usepackage{subfig}

% Include other packages here, before hyperref.

% If you comment hyperref and then uncomment it, you should delete
% egpaper.aux before re-running latex.  (Or just hit 'q' on the first latex
% run, let it finish, and you should be clear).
\usepackage[pagebackref=true,breaklinks=true,letterpaper=true,colorlinks,bookmarks=false]{hyperref}

% ============================================================================================
% COMMENTS
% ============================================================================================
\newcommand{\stavros}[1]{{\textcolor{red}{[\emph{stavros}: #1]}}}

% ============================================================================================
% GENERAL DEFINITIONS
% ============================================================================================
\usepackage{xspace}
\def\sota{state-of-the-art}
\def\groundtruth{ground truth}
\def\CUB{CUB-200-2011}
\def\eg{\emph{e.g}\onedot} \def\Eg{\emph{E.g}\onedot}
\def\ie{\emph{i.e}\onedot} \def\Ie{\emph{I.e}\onedot}
\def\cf{\emph{c.f}\onedot} \def\Cf{\emph{C.f}\onedot}
\def\etc{\emph{etc}\onedot} \def\vs{\emph{vs}\onedot}
\def\wrt{w.r.t\onedot} \def\dof{d.o.f\onedot}
\def\etal{\emph{et al}\onedot}
\def\kmeans{\emph{k}-means}
\def\cpp{C\texttt{++}}
\def\matlab{MATLAB}
\def\matconvnet{MatConvNet}

% Chapters, Sections, Figures, Equations
\newcommand{\refchap}[1]{Chapter~\ref{#1}}
\newcommand{\refsec}[1]{Section~\ref{#1}}
\newcommand{\reffig}[1]{Figure~\ref{#1}}
\newcommand{\refeq}[1]{Equation~\ref{#1}}
\newcommand{\reftab}[1]{Table~\ref{#1}}
\newcommand{\refapp}[1]{Appendix~\ref{#1}}

% ============================================================================================
% MATH DEFINITIONS 
% ============================================================================================
\newcommand{\pd}[2]{\frac{\partial #1}{\partial #2}}
\newcommand{\mat}[1]{\mathbf{#1}} 
\renewcommand{\vec}[1]{\mathbf{#1}}
\newcommand{\set}[1]{\mathcal{#1}}
\newcommand{\inprod}[1]{{\left< \, #1 \, \right>}}
\newcommand{\R}{\mathbb{R}}
\def\abs{\operatorname{abs}}
\DeclareMathOperator*{\argmin}{argmin}
\DeclareMathOperator*{\argmax}{argmax}
\newcommand{\p}[1]{\vec{p}_{#1}} 	 % point with subscript
\newcommand{\norm}[1]{\left \lVert #1 \right \rVert} % norm


% \iccvfinalcopy % *** Uncomment this line for the final submission

\def\iccvPaperID{1142} % *** Enter the ICCV Paper ID here
\def\httilde{\mbox{\tt\raisebox{-.5ex}{\symbol{126}}}}

\begin{document}
\title{AMAT: Medial Axis Transform for Natural Images \\ Supplemental Material}
\maketitle
%\thispagestyle{empty}

\section{Introduction}\label{sec:introduction}
In this supplemental material we include details and qualitative results that we could not include in the main submission due to the limited space.
\stavros{Complete depending on the material we add. Don't forget to include the video.}


\section{Complexity Analysis and How}\label{sec:complexity}
First, we re-define some useful quantities to make the supplemental material self-contained.
We consider a finite set of radii $r\in \scales : \{r_1,\ldots,r_R \}$, and a disk of radius $r_j$, centered at point $\p{i}\in X^I$,
is denoted as $D_{\p{i},r_j} = D_{ij}$.
The most computationally demanding step in our method is the computation of the disk costs
\begin{equation}
c_{ij} = \sum_k \sum_l \norm{\f{ij} - \f{kl} }^2 \quad \forall k,l: D_{kl} \subset D_{ij}.
\label{eq:diskcost}
\end{equation}
This quantity must be computed at all location-radius combinations, and for all the contained disks, of radii $r\in\scales$.

To simplify computations, we consider the continuous case.




\section{Qualitative Results}\label{sec:qualitative}

\


{\small
\bibliographystyle{ieee}
\bibliography{iccv2017.bib}
}

\end{document}