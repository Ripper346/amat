\documentclass[10pt,twocolumn,letterpaper]{article}
% PACKAGES ===================================================================================
\usepackage{iccv}
\usepackage{times}
\usepackage{epsfig}
\usepackage{graphicx}
\usepackage{amsmath,amssymb}
\usepackage{algorithm,algpseudocode}
\usepackage{subfig}

% Include other packages here, before hyperref.

% If you comment hyperref and then uncomment it, you should delete
% egpaper.aux before re-running latex.  (Or just hit 'q' on the first latex
% run, let it finish, and you should be clear).
\usepackage[pagebackref=true,breaklinks=true,letterpaper=true,colorlinks,bookmarks=false]{hyperref}

% ============================================================================================
% COMMENTS
% ============================================================================================
\newcommand{\stavros}[1]{{\textcolor{red}{[\emph{stavros}: #1]}}}

% ============================================================================================
% GENERAL DEFINITIONS
% ============================================================================================
\usepackage{xspace}
\def\sota{state-of-the-art}
\def\groundtruth{ground truth}
\def\CUB{CUB-200-2011}
\def\eg{\emph{e.g}\onedot} \def\Eg{\emph{E.g}\onedot}
\def\ie{\emph{i.e}\onedot} \def\Ie{\emph{I.e}\onedot}
\def\cf{\emph{c.f}\onedot} \def\Cf{\emph{C.f}\onedot}
\def\etc{\emph{etc}\onedot} \def\vs{\emph{vs}\onedot}
\def\wrt{w.r.t\onedot} \def\dof{d.o.f\onedot}
\def\etal{\emph{et al}\onedot}
\def\kmeans{\emph{k}-means}
\def\cpp{C\texttt{++}}
\def\matlab{MATLAB}
\def\matconvnet{MatConvNet}

% Chapters, Sections, Figures, Equations
\newcommand{\refchap}[1]{Chapter~\ref{#1}}
\newcommand{\refsec}[1]{Section~\ref{#1}}
\newcommand{\reffig}[1]{Figure~\ref{#1}}
\newcommand{\refeq}[1]{Equation~\ref{#1}}
\newcommand{\reftab}[1]{Table~\ref{#1}}
\newcommand{\refapp}[1]{Appendix~\ref{#1}}

% ============================================================================================
% MATH DEFINITIONS 
% ============================================================================================
\newcommand{\pd}[2]{\frac{\partial #1}{\partial #2}}
\newcommand{\mat}[1]{\mathbf{#1}} 
\renewcommand{\vec}[1]{\mathbf{#1}}
\newcommand{\set}[1]{\mathcal{#1}}
\newcommand{\inprod}[1]{{\left< \, #1 \, \right>}}
\newcommand{\R}{\mathbb{R}}
\def\abs{\operatorname{abs}}
\DeclareMathOperator*{\argmin}{argmin}
\DeclareMathOperator*{\argmax}{argmax}
\newcommand{\p}[1]{\vec{p}_{#1}} 	 % point with subscript
\newcommand{\norm}[1]{\left \lVert #1 \right \rVert} % norm


\def\iccvPaperID{1142} % *** Enter the ICCV Paper ID here
\def\httilde{\mbox{\tt\raisebox{-.5ex}{\symbol{126}}}}

\begin{document}
\title{AMAT: Medial Axis Transform for Natural Images \\ Supplemental Material}
\maketitle
%\thispagestyle{empty}

\section{Introduction}\label{sec:introduction}
In this supplemental material we include details and qualitative results that we could not include in the main submission due to the limited space.
We analytically derive the computational complexity of the disk cost computation, which is the most demanding step in our method, 
and discuss ways of reducing it.
We also present additional qualitative results of medial axes computed with our method, and the respective reconstructions.
Finally, in the zip file for the supplemental material we also include a video showing the execution of the greedy algorithm on a test image.
In the video we visualize the process of selecting disks and covering the image, as well as the medial axis construction and respective radii.

\section{Complexity Analysis}\label{sec:complexity}
First, we remind the notation for some useful quantities to make the supplemental material self-contained.
A disk of radius $r_j\in \scales = \{r_1,\ldots,r_R\}$, 
centered at point $\p{i}\in X^I \subset \R^2$, is denoted as $D_{\p{i},r_j} = D_{ij}$.
We also assume that $X^I$ is a universe of $N$ points in total, corresponding to the spatial support of an input RGB image $I$.
The most demanding step in our method is the computation of the disk costs
\begin{equation}
c_{ij} = \sum_k \sum_l \norm{\f{ij} - \f{kl} }^2 \quad \forall k,l: D_{kl} \subset D_{ij}.
\label{eq:diskcost}
\end{equation}
Simply computing the encodings $\f{ij}$ at all possible locations and scales has complexity $O(NR^3)$.
However, this quantity must be computed at all location-radius combinations, and \emph{for all the contained disks} of radii $r\in\scales$.

For simplicity, we make all derivations in the continuous domain, but they can be readily extended to the discrete domain.
Let $r,r'\in(0,R]$, with $r<r'$, and consider a disk of radius $r'$, centered at point $\p{}'$, $D_{\p{}',r'}$.
A \emph{fully contained} disk $D_{\p{},r} \subset D_{\p{}',r'}$ can be placed anywhere inside the disk
$(x-x_{\p{}'})^2 + (y-y_{\p{}'})^2 \leq (r'-r)^2$.
It follows that the number of such contained disks for a given radius $r'$ is
\begin{equation}
N_d^{r'} = \int_{0}^{r'}  \pi (r'-r)^2 dr,\label{eq:ndr}
\end{equation}
and summing over all possible radii $r'\in(0,R]$ we have that the total number of disks at $\p{}$ is
\begin{equation}
N_d = \int_0^R N_d^{r'}dr'.\label{eq:nd}
\end{equation}
Putting together~\refeq{eq:ndr} and~\refeq{eq:nd} we get:
\begin{align}
\nonumber
N_d & = \pi \int_0^{R} \int_0^{r'} (r'-r)^2 dr dr' \\ \nonumber
  & = \pi \int_0^{R} \int_0^{r'} (r')^2 -2r'r + r^2 dr dr' \\ \nonumber
  & = \pi \int_0^{R} \left. (r')^2 r - r'r^2 + \frac{r^3}{3} \right \rvert_{0}^{r'} dr' \\ 
  & = \pi \int_0^{R}  \frac{(r')^3}{3} dr' = \pi \left. \frac{(r')^4}{12}\right \rvert_0^R  = \frac{\pi}{12} R^4. 
\end{align}
This is the (approximate up to a constant factor) number of contained disks and respective encodings that must be considered to compute $c_{ij}$ at
all possible scales $r \in (0,R]$, at a single point in the image.
Repeating this process for all $N$ points in the image domain yields a total complexity of $O(NR^4)$.

\subsection{Reducing Complexity}\label{sec:complexity:reducing}
The $O(NR^4)$ complexity derived in the previous section can reduced through a combination of the following: 
i) rewriting~\refeq{eq:diskcost} in an equivalent form that is amenable to efficient computation using convolutions; 
ii) using cumulative sums to avoid redundant computations.
Since we will release our code, we do not elaborate further on i).

We provide an intuitive explanation for ii):
Suppose we denote by $\sum_{D_{ij}} \f{}$ the sum of a quantity $\f{}$ within a disk area $D_{ij}$ of radius $r_j$.
Naively calculating this quantity for a disk of radius $r_l > r_j$ involves redundant re-computations of $\f{}$
in the area $D_{ij} \subset D_{il}$.
We can avoid these redundant computations, observing that 
\begin{equation}
\sum_{D_{ij}} \f{} = \sum_{j=1}^R \sum_{C_{ij}} \f{}. \label{eq:circlesum}
\end{equation}
Using the relationship in~\refeq{eq:circlesum} we reduce the complexity of computing feature encodings at all locations from
$O(NR^3)$ to $O(NR^2)$ and the complexity of disk cost computation from $O(NR^4)$ to $O(NR^3)$.

Using a shape other than disks in our approach opens up the possibility of further reducing complexity.
For example, one can consider using rectangles or squares to approximate or replace disks, 
as in~\cite{arbelaez2011contour,tsogkas2012learning}.
In that scenario, pre-computing an integral image would allow us to use a fixed number of operations per image point, further
reducing complexity of encoding and disk cost computation to $O(NR)$ and $O(NR^2)$ respectively.
We aim to explore this in future work.



%We start by rewriting~\refeq{eq:diskcost} in the following form:
%\begin{align}
%\nonumber
%c_{ij} & = \sum_k \sum_l \norm{\f{ij} - \f{kl} }^2 = \sum_k \sum_l (\f{ij} - \f{kl} )^2 \\ \nonumber
%	   & = \f{ij}^2 \sum_k \sum_l 1  + \sum_k \sum_l \f{kl}^2 -2\f{ij}\sum_k \sum_l \f{kl} \\ 
%	   & = \f{ij}^2 N_d^{r_j}  + \sum_k \sum_l \f{kl}^2 -2\f{ij}\sum_k \sum_l \f{kl}  
%\end{align}
%The first term in equation



\begin{figure*}[t]
\centering
\def\imgw{0.245}

\def\img_id{85048}
\includegraphics[width=\imgw \textwidth]{\img_id_resized.jpg}
\includegraphics[width=\imgw \textwidth]{\img_id_axes_simplified.pdf}
\includegraphics[width=\imgw \textwidth]{\img_id_branches_simplified.pdf}
\includegraphics[width=\imgw \textwidth]{\img_id_gt_skel.pdf}

\def\img_id{295087}
\includegraphics[width=\imgw \textwidth]{\img_id_resized.jpg}
\includegraphics[width=\imgw \textwidth]{\img_id_axes_simplified.pdf}
\includegraphics[width=\imgw \textwidth]{\img_id_branches_simplified.pdf}
\includegraphics[width=\imgw \textwidth]{\img_id_gt_skel.pdf}

\def\img_id{145086}
\includegraphics[width=\imgw \textwidth]{\img_id_resized.jpg}
\includegraphics[width=\imgw \textwidth]{\img_id_axes_simplified.pdf}
\includegraphics[width=\imgw \textwidth]{\img_id_branches_simplified.pdf}
\includegraphics[width=\imgw \textwidth]{\img_id_gt_skel.pdf}

\def\img_id{101087}
\includegraphics[width=\imgw \textwidth]{\img_id_resized.jpg}
\includegraphics[width=\imgw \textwidth]{\img_id_axes_simplified.pdf}
\includegraphics[width=\imgw \textwidth]{\img_id_branches_simplified.pdf}
\includegraphics[width=\imgw \textwidth]{\img_id_gt_skel.pdf}

\def\img_id{54082}
\includegraphics[width=\imgw \textwidth]{\img_id_resized.jpg}
\includegraphics[width=\imgw \textwidth]{\img_id_axes_simplified.pdf}
\includegraphics[width=\imgw \textwidth]{\img_id_branches_simplified.pdf}
\includegraphics[width=\imgw \textwidth]{\img_id_gt_skel.pdf}
\caption{\textbf{Medial axes}. From left to right: Input image, AMAT medial axes, medial branches (color-coded), ground-truth skeletons.
Axis color indicates the respective encodings $\f{}$, and black is used for unused points.}
\label{fig:axes}
\end{figure*}



\begin{figure*}[t]
\centering
\def\imgw{0.195}

\def\img_id{3096}
\includegraphics[width=\imgw \textwidth]{\img_id_resized.jpg}
\includegraphics[width=\imgw \textwidth]{\img_id_rec_mil.png}
\includegraphics[width=\imgw \textwidth]{\img_id_rec_gtseg.png}
\includegraphics[width=\imgw \textwidth]{\img_id_rec_gtskel.png}
\includegraphics[width=\imgw \textwidth]{\img_id_rec_amat.png}

\def\img_id{300091}
\includegraphics[width=\imgw \textwidth]{\img_id_resized.jpg}
\includegraphics[width=\imgw \textwidth]{\img_id_rec_mil.png}
\includegraphics[width=\imgw \textwidth]{\img_id_rec_gtseg.png}
\includegraphics[width=\imgw \textwidth]{\img_id_rec_gtskel.png}
\includegraphics[width=\imgw \textwidth]{\img_id_rec_amat.png}

\def\img_id{119082}
\includegraphics[width=\imgw \textwidth]{\img_id_resized.jpg}
\includegraphics[width=\imgw \textwidth]{\img_id_rec_mil.png}
\includegraphics[width=\imgw \textwidth]{\img_id_rec_gtseg.png}
\includegraphics[width=\imgw \textwidth]{\img_id_rec_gtskel.png}
\includegraphics[width=\imgw \textwidth]{\img_id_rec_amat.png}

\def\img_id{295087}
\includegraphics[width=\imgw \textwidth]{\img_id_resized.jpg}
\includegraphics[width=\imgw \textwidth]{\img_id_rec_mil.png}
\includegraphics[width=\imgw \textwidth]{\img_id_rec_gtseg.png}
\includegraphics[width=\imgw \textwidth]{\img_id_rec_gtskel.png}
\includegraphics[width=\imgw \textwidth]{\img_id_rec_amat.png}

\def\img_id{54082}
\includegraphics[width=\imgw \textwidth]{\img_id_resized.jpg}
\includegraphics[width=\imgw \textwidth]{\img_id_rec_mil.png}
\includegraphics[width=\imgw \textwidth]{\img_id_rec_gtseg.png}
\includegraphics[width=\imgw \textwidth]{\img_id_rec_gtskel.png}
\includegraphics[width=\imgw \textwidth]{\img_id_rec_amat.png}

\def\img_id{101087}
\includegraphics[width=\imgw \textwidth]{\img_id_resized.jpg}
\includegraphics[width=\imgw \textwidth]{\img_id_rec_mil.png}
\includegraphics[width=\imgw \textwidth]{\img_id_rec_gtseg.png}
\includegraphics[width=\imgw \textwidth]{\img_id_rec_gtskel.png}
\includegraphics[width=\imgw \textwidth]{\img_id_rec_amat.png}

\caption{\textbf{Image reconstruction}. From left to right: Input image, MIL~\cite{tsogkas2012learning}, GT-seg, GT-skel, AMAT.}
\label{fig:experiments:reconstruction}
\end{figure*}





{\small
\bibliographystyle{ieee}
\bibliography{iccv2017.bib}
}

\end{document}