% PACKAGES ===================================================================================
\usepackage{iccv}
\usepackage{times}
\usepackage{epsfig}
\usepackage{graphicx}
\usepackage{amsmath}
\usepackage{amssymb}
\usepackage{algorithm}
\usepackage[noend]{algorithmic}
\usepackage{subfig}

% Include other packages here, before hyperref.

% If you comment hyperref and then uncomment it, you should delete
% egpaper.aux before re-running latex.  (Or just hit 'q' on the first latex
% run, let it finish, and you should be clear).
\usepackage[pagebackref=true,breaklinks=true,letterpaper=true,colorlinks,bookmarks=false]{hyperref}

% ============================================================================================
% COMMENTS
% ============================================================================================
\newcommand{\stavros}[1]{{\textcolor{red}{[\emph{stavros}: #1]}}}

% ============================================================================================
% GENERAL DEFINITIONS
% ============================================================================================
\usepackage{xspace}
\def\sota{state-of-the-art}
\def\groundtruth{ground truth}
\def\CUB{CUB-200-2011}
\def\eg{\emph{e.g}\onedot} \def\Eg{\emph{E.g}\onedot}
\def\ie{\emph{i.e}\onedot} \def\Ie{\emph{I.e}\onedot}
\def\cf{\emph{c.f}\onedot} \def\Cf{\emph{C.f}\onedot}
\def\etc{\emph{etc}\onedot} \def\vs{\emph{vs}\onedot}
\def\wrt{w.r.t\onedot} \def\dof{d.o.f\onedot}
\def\etal{\emph{et al}\onedot}
\def\kmeans{\emph{k}-means}
\def\cpp{C\texttt{++}}
\def\matlab{MATLAB}
\def\matconvnet{MatConvNet}

% Chapters, Sections, Figures, Equations
\newcommand{\refchap}[1]{Chapter~\ref{#1}}
\newcommand{\refsec}[1]{Section~\ref{#1}}
\newcommand{\reffig}[1]{Figure~\ref{#1}}
\newcommand{\refeq}[1]{Equation~\ref{#1}}
\newcommand{\reftab}[1]{Table~\ref{#1}}
\newcommand{\refapp}[1]{Appendix~\ref{#1}}

% ============================================================================================
% MATH DEFINITIONS 
% ============================================================================================
\newcommand{\pd}[2]{\frac{\partial #1}{\partial #2}}
\newcommand{\mat}[1]{\mathbf{#1}} 
\renewcommand{\vec}[1]{\mathbf{#1}}
\newcommand{\set}[1]{\mathcal{#1}}
\newcommand{\inprod}[1]{{\left< \, #1 \, \right>}}
\newcommand{\R}{\mathbb{R}}
\def\abs{\operatorname{abs}}
\DeclareMathOperator*{\argmin}{argmin}
\DeclareMathOperator*{\argmax}{argmax}
\newcommand{\D}[2]{D_{\vec{#1},#2}} % (p,r)-disk
\newcommand{\Di}[3]{D_{\vec{#1}_#3,#2_#3}} % (p,r)-disk with subscript
\newcommand{\Dg}[2]{\tilde{D}_{\vec{#1},#2}} % (p,r)-disk decoding
\newcommand{\Dgi}[3]{\tilde{D}_{\vec{#1}_#3,#2_#3}} % (p,r)-disk decoding
\newcommand{\norm}[1]{\left \lVert #1 \right \rVert} % norm
