\documentclass[10pt,twocolumn,letterpaper]{article}

% PACKAGES ===================================================================================
\usepackage{iccv}
\usepackage{times}
\usepackage{epsfig}
\usepackage{graphicx}
\usepackage{amsmath,amssymb}
\usepackage{algorithm,algpseudocode}
\usepackage{subfig}

% Include other packages here, before hyperref.

% If you comment hyperref and then uncomment it, you should delete
% egpaper.aux before re-running latex.  (Or just hit 'q' on the first latex
% run, let it finish, and you should be clear).
\usepackage[pagebackref=true,breaklinks=true,letterpaper=true,colorlinks,bookmarks=false]{hyperref}

% ============================================================================================
% COMMENTS
% ============================================================================================
\newcommand{\stavros}[1]{{\textcolor{red}{[\emph{stavros}: #1]}}}

% ============================================================================================
% GENERAL DEFINITIONS
% ============================================================================================
\usepackage{xspace}
\def\sota{state-of-the-art}
\def\groundtruth{ground truth}
\def\CUB{CUB-200-2011}
\def\eg{\emph{e.g}\onedot} \def\Eg{\emph{E.g}\onedot}
\def\ie{\emph{i.e}\onedot} \def\Ie{\emph{I.e}\onedot}
\def\cf{\emph{c.f}\onedot} \def\Cf{\emph{C.f}\onedot}
\def\etc{\emph{etc}\onedot} \def\vs{\emph{vs}\onedot}
\def\wrt{w.r.t\onedot} \def\dof{d.o.f\onedot}
\def\etal{\emph{et al}\onedot}
\def\kmeans{\emph{k}-means}
\def\cpp{C\texttt{++}}
\def\matlab{MATLAB}
\def\matconvnet{MatConvNet}

% Chapters, Sections, Figures, Equations
\newcommand{\refchap}[1]{Chapter~\ref{#1}}
\newcommand{\refsec}[1]{Section~\ref{#1}}
\newcommand{\reffig}[1]{Figure~\ref{#1}}
\newcommand{\refeq}[1]{Equation~\ref{#1}}
\newcommand{\reftab}[1]{Table~\ref{#1}}
\newcommand{\refapp}[1]{Appendix~\ref{#1}}

% ============================================================================================
% MATH DEFINITIONS 
% ============================================================================================
\newcommand{\pd}[2]{\frac{\partial #1}{\partial #2}}
\newcommand{\mat}[1]{\mathbf{#1}} 
\renewcommand{\vec}[1]{\mathbf{#1}}
\newcommand{\set}[1]{\mathcal{#1}}
\newcommand{\inprod}[1]{{\left< \, #1 \, \right>}}
\newcommand{\R}{\mathbb{R}}
\def\abs{\operatorname{abs}}
\DeclareMathOperator*{\argmin}{argmin}
\DeclareMathOperator*{\argmax}{argmax}
\newcommand{\p}[1]{\vec{p}_{#1}} 	 % point with subscript
\newcommand{\norm}[1]{\left \lVert #1 \right \rVert} % norm


% \iccvfinalcopy % *** Uncomment this line for the final submission

\def\iccvPaperID{****} % *** Enter the ICCV Paper ID here
\def\httilde{\mbox{\tt\raisebox{-.5ex}{\symbol{126}}}}

\begin{document}
\title{\LaTeX\ Author Guidelines for ICCV Proceedings}
\maketitle
%\thispagestyle{empty}


%%%%%%%%% ABSTRACT
\begin{abstract}
\end{abstract}

%%%%%%%%% BODY TEXT %%%%%%%%%%%%%%%%%%%%%%%%%%%%%%%%%%
% ==========================================================
\section{Introduction}\label{sec:introduction}
% ==========================================================
\begin{itemize}
\item MAT, perceptual grouping, symmetry.
\item application in which MAT has been used (shape matching, shape retrieval etc)
\item
\end{itemize}


% ==========================================================
\section{Related Work}\label{sec:related}
% ==========================================================
\subsubsection*{MAT for binary shapes}
\begin{itemize}
\item Blum's Medial axis transform
\item Other works on extracting skeletons for 2D and 3D shapes (shock graphs,bone graphs,scale axis transform,check out other graphics works)
\end{itemize}
\subsubsection*{MAT for natural images}
\begin{itemize}
\item Levinstein's work
\item MIL-based symmetry detection
\item Random-forest variant by Alloimonos' student
\item CNN-variant 
\end{itemize}

% ==========================================================
\section{Background}\label{sec:background}
% ==========================================================
To make the paper self-contained, we include a brief review of two basic concepts that our work is based on.

\subsubsection*{Medial Axis Transform}
Consider a two-dimensional binary shape, $O$, in the image plane, and its boundary $\Theta_O$.
The \emph{medial axis} of $O$ is the set of points $\vec{p}$ that lie midway between two sections of 
its boundary. The points $\vec{p}$ are the centers of maximally inscribed disks, bitangent to $\Theta_O$
in the interior of the shape. The radius $r(\vec{p})$ of the disk is the distance between $\vec{p}$ and 
the points where the disk touches $\Theta_O$, and is called the \emph{medial radius}.
Based on the above definitions, the \emph{medial axis transform} is the process of 
mapping $O$ to the set of tuples $(\vec{p},r(\vec{p})) \in \mathbb{R}^2 \times \mathbb{R}$.
Given these pairs, we can fully reconstruct the object as a union of overlapping disks that sweep-out 
the object's interior. 

\subsubsection*{Weighted Geometric Set Cover}
The geometric set cover is the extension of the well studied set cover problem, in a geometric space.
For simplicity, here we only consider the case of a two-dimensional space and we particularly focus in the 
\emph{weighted} version of the problem, which is defined as follows:
Consider a universe of $n$ points $set{X} \in \R^2$ and a family of sets 
$\set{D} = \{D_1,D_2,\ldots,D_m\} \subseteq \set{X}$. 
A common choice for these sets are intersections of $\set{X}$ and simple shape primitives, such as disks or rectangles.

Now assume that each set in $\set{D}$ is associated with a non-negative weight or \emph{cost} $c_i$.
Solving the WGSC problem amounts to finding a sub-collection $\bar{\set{D}} \in \set{D}$ that covers the entire $\set{X}$
(all $n$ elements of $\set{X}$ are contained in at least one set in $\bar{\set{D}}$), while having the minimum
total cost $C$; the total cost is simply the sum of costs of individual elements in $\bar{\set{D}}$.

WGSC is an NP-hard problem for which several polynomial-time approximate solutions (PTAS) exist~\cite{PTAS algos for WGSC}.
Perhaps the simplest and more straightforward is the greedy algorithm described in~\cite{Vazirani}. 
\stavros{We will include pseudocode for the algorithm applied in our own problem}
The interested reader can find more details on the WGSC and related algorithms in~\cite{other WGSC works}.


% ==========================================================
\section{Method}\label{sec:method}
% ==========================================================
\subsection{Problem Definition}\label{sec:definition}

\subsection{AMAT as a Geometric Set Cover Problem}

\subsection{Extending AMAT to Textured Images}
\begin{itemize}
\item $L_0$ Smoothing
\item GANs (if that works)
\end{itemize}

% ==========================================================
\section{Experiments}\label{sec:experiments}
% ==========================================================
\subsection{Image Reconstruction}
Experiments on image reconstruction using standard metrics (MSE, PSNR, and metrics using CNN feature similarity.

\subsubsection*{Abstract Scenes}

\subsubsection*{BSDS500}
This can only be done if we can make the GAN work and generate texture.

\subsection{Medial Point Detection (BMAX500)}\label{sec:medial-point-detection}
Recover all medial axes of the BSDS500. Compare with other skeletonization algorithms, such
as the CNN based method, and my previous MIL-based method

\subsection{Boundary Detection (BSDS500)}\label{sec:boundary-detection}
Medial axes to boundaries. 
MIL: compute an approximate scale based on the scale with maximum probability and try to recover boundaries.
CNN: same as above, this was already done in that paper for estimating segmentation masks using scale information.
We can compare the boundary recovery results as well.
EDGE DETECTORS: Standard edge detection algorithms (Dollar, HED, Canny, gPB etc: compare with algorithms specifically
tailored for this task. WE ARE NOT EXPECTING SOTA PERFORMANCE.

\subsection{Object Proposals}\label{sec:object-proposals}

% ==========================================================
\section{Discussion}\label{sec:discussion}
% ==========================================================

{\small
\bibliographystyle{ieee}
\bibliography{egbib}
}

\end{document}
