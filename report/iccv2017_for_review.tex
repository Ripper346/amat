\documentclass[10pt,twocolumn,letterpaper]{article}

% PACKAGES ===================================================================================
\usepackage{iccv}
\usepackage{times}
\usepackage{epsfig}
\usepackage{graphicx}
\usepackage{amsmath,amssymb}
\usepackage{algorithm,algpseudocode}
\usepackage{subfig}

% Include other packages here, before hyperref.

% If you comment hyperref and then uncomment it, you should delete
% egpaper.aux before re-running latex.  (Or just hit 'q' on the first latex
% run, let it finish, and you should be clear).
\usepackage[pagebackref=true,breaklinks=true,letterpaper=true,colorlinks,bookmarks=false]{hyperref}

% ============================================================================================
% COMMENTS
% ============================================================================================
\newcommand{\stavros}[1]{{\textcolor{red}{[\emph{stavros}: #1]}}}

% ============================================================================================
% GENERAL DEFINITIONS
% ============================================================================================
\usepackage{xspace}
\def\sota{state-of-the-art}
\def\groundtruth{ground truth}
\def\CUB{CUB-200-2011}
\def\eg{\emph{e.g}\onedot} \def\Eg{\emph{E.g}\onedot}
\def\ie{\emph{i.e}\onedot} \def\Ie{\emph{I.e}\onedot}
\def\cf{\emph{c.f}\onedot} \def\Cf{\emph{C.f}\onedot}
\def\etc{\emph{etc}\onedot} \def\vs{\emph{vs}\onedot}
\def\wrt{w.r.t\onedot} \def\dof{d.o.f\onedot}
\def\etal{\emph{et al}\onedot}
\def\kmeans{\emph{k}-means}
\def\cpp{C\texttt{++}}
\def\matlab{MATLAB}
\def\matconvnet{MatConvNet}

% Chapters, Sections, Figures, Equations
\newcommand{\refchap}[1]{Chapter~\ref{#1}}
\newcommand{\refsec}[1]{Section~\ref{#1}}
\newcommand{\reffig}[1]{Figure~\ref{#1}}
\newcommand{\refeq}[1]{Equation~\ref{#1}}
\newcommand{\reftab}[1]{Table~\ref{#1}}
\newcommand{\refapp}[1]{Appendix~\ref{#1}}

% ============================================================================================
% MATH DEFINITIONS 
% ============================================================================================
\newcommand{\pd}[2]{\frac{\partial #1}{\partial #2}}
\newcommand{\mat}[1]{\mathbf{#1}} 
\renewcommand{\vec}[1]{\mathbf{#1}}
\newcommand{\set}[1]{\mathcal{#1}}
\newcommand{\inprod}[1]{{\left< \, #1 \, \right>}}
\newcommand{\R}{\mathbb{R}}
\def\abs{\operatorname{abs}}
\DeclareMathOperator*{\argmin}{argmin}
\DeclareMathOperator*{\argmax}{argmax}
\newcommand{\p}[1]{\vec{p}_{#1}} 	 % point with subscript
\newcommand{\norm}[1]{\left \lVert #1 \right \rVert} % norm


% \iccvfinalcopy % *** Uncomment this line for the final submission

\def\iccvPaperID{****} % *** Enter the ICCV Paper ID here
\def\httilde{\mbox{\tt\raisebox{-.5ex}{\symbol{126}}}}

\begin{document}
\title{\LaTeX\ Author Guidelines for ICCV Proceedings}
\maketitle
%\thispagestyle{empty}


%%%%%%%%% ABSTRACT
\begin{abstract}
\end{abstract}

%%%%%%%%% BODY TEXT %%%%%%%%%%%%%%%%%%%%%%%%%%%%%%%%%%
% ==========================================================
\section{Introduction}\label{sec:introduction}
% ==========================================================
\begin{itemize}
\item MAT, perceptual grouping, symmetry.
\item application in which MAT has been used (shape matching, shape retrieval etc)
\item
\end{itemize}


% ==========================================================
\section{Related Work}\label{sec:related}
% ==========================================================
\subsubsection*{MAT for binary shapes}
\begin{itemize}
\item Blum's Medial axis transform
\item Other works on extracting skeletons for 2D and 3D shapes (shock graphs,bone graphs,scale axis transform,check out other graphics works)
\end{itemize}
\subsubsection*{MAT for natural images}
\begin{itemize}
\item Levinstein's work
\item MIL-based symmetry detection
\item Random-forest variant by Alloimonos' student
\item CNN-variant 
\end{itemize}

% ==========================================================
\section{Background}\label{sec:background}
% ==========================================================
\stavros{Perhaps move this section in~\refsec{sec:method}}
To make the paper self-contained, we include a brief review of two basic concepts that our work is based on.

\subsubsection*{Medial Axis Transform}
Consider a two-dimensional binary shape, $O$, in the image plane, and its boundary $\Theta_O$.
The \emph{medial axis} of $O$ is the set of points $\vec{p}$ that lie midway between two sections of 
its boundary. The points $\vec{p}$ are the centers of maximally inscribed (medial) disks, bitangent to $\Theta_O$
in the interior of the shape. The radius $r(\vec{p})$ of the disk is the distance between $\vec{p}$ and 
the points where the disk touches $\Theta_O$, and is called the \emph{medial radius}.
Based on the above definitions, the \emph{medial axis transform} is the process of 
mapping $O$ to the set of tuples $(\vec{p},r(\vec{p})) \in \mathbb{R}^2 \times \mathbb{R}$.
Given these pairs, we can reconstruct the object as a union of overlapping disks that sweep-out 
its interior. 

\subsubsection*{Weighted Geometric Set Cover}
The geometric set cover is the extension of the well studied set cover problem, in a geometric space.
For simplicity, here we only consider the case of a two-dimensional space and we particularly focus in the 
\emph{weighted} version of the problem, which is defined as follows:
Consider a universe of $n$ points $\set{X} \in \R^2$ and a family of sets 
$\set{D} = \{D_1,D_2,\ldots,D_m\} \subseteq \set{X}$. 
A common choice for these sets are intersections of $\set{X}$ with simple shape primitives, such as disks or rectangles.

Now assume that each set in $\set{D}$ is associated with a non-negative weight or \emph{cost} $c_i$.
Solving the WGSC problem amounts to finding a sub-collection $\bar{\set{D}} \in \set{D}$ that covers the entire $\set{X}$
(all $n$ elements of $\set{X}$ are contained in at least one set in $\bar{\set{D}}$), while having the minimum
total cost $C$; the total cost is simply the sum of costs of individual elements in $\bar{\set{D}}$.

WGSC is an NP-hard problem for which polynomial-time approximate solutions (PTAS) exist.
Perhaps the simplest and more straightforward is the greedy algorithm described in~\cite{vazirani2013approximation}. 
\stavros{We will include pseudocode for the algorithm applied in our own problem}
The interested reader can find more details on WGSC and related algorithms in~\cite{mustafa2015quasi,varadarajan2010weighted,har2012weighted,chan2012weighted}.


% ==========================================================
\section{Method}\label{sec:method}
% ==========================================================
\subsection{Problem Definition}\label{sec:definition}
Our definition of a medial axis transform for color images is inspired by the invertibility property of the binary MAT. 
Given a medial point $\vec{p}_i$ and its associated scale $r_i$, we can reconstruct part of the shape by ``expanding'' a value of one (1) inside the area covered by the medial disk or radius $r_i$, centered at $\vec{p}_i$.
Repeating this process for all points lying on the medial axis, results in a reconstruction of the full shape.

We argue that a MAT for real images should satisfy a similar principle: given the MAT of an image, we should be able to ``invert'' it, reconstructing \emph{the image} itself.
There are several reasons why extending this idea from binary to real images is a challenging task. 
Natural images contain complex scenes and are cluttered with various objects, instead of a single foreground shape. 
Moreover, whereas binary images are composed of uniform areas of zeros (0) and ones (1), real images obey complicated color and texture distributions. Nevertheless, we can assume that in most cases an image is composed of many small regions of relatively uniform appearance. This is a concept on which most superpixel algorithms are built on: group pixels into non-overlapping patches that capture image redundancies, while respecting perceptually meaningful region boundaries~\cite{shi2000normalized,levinshtein2009turbopixels,achanta2012slic}. 

\paragraph{Notation:} Before proceeding with the details of our formulation, we list the notation used throughout our paper.
We use capital italics for single elements and capital calligraphic letters to denote sets. A disk of radius $r$, centered at point $\vec{p}$ is denoted as $D(\vec{p},r)=\D{p}{r}$. For brevity, we ofter refer to such a disk as a $r$-disk or $(\vec{p},r)$-disk.
The interesection of a $(\vec{p},r)$-disk with and image $I$ is a disk-shape region of the image, and is denoted by 
$I \cap \D{p}{r}=\D{p}{r}^I \subset I$. Finally, we use $\circ$ to denote function composition.

\paragraph{Formulation:} Consider an RGB image $I\subset\R^2$, and a disk-shaped region $\D{p}{r}^I \subset I$.
Let $f$ be a function that maps $\D{p}{r}^I$ to a vector representation $f_{\vec{p},r}=f\circ \D{p}{r}^I$. 
We call this representation the \emph{encoding} of $\D{p}{r}^I$. 
Now let $g$ be a function that maps this encoding back to a disk patch $\Dg{p}{r}^I$: 
$g_{\vec{p},r} = g \circ f_{\vec{p},r} = \Dg{p}{r}^I$. We call $g$ the \emph{decoding} function.
In the general case, $f$ and $g$ will be \emph{lossy} mappings, which means that the reconstruction error 
$e_{\vec{p},r} = \norm{ \Dg{p}{r}^I - \D{p}{r}^I}_2 \geq 0$. 

Using the above ideas, we define MAT as the set of tuples 
$M:\{(\vec{p}_1,r_1,f_{\vec{p}_1,r_1}),\ldots,\vec{p}_m,r_m,f_{\vec{p}_m,r_m})\}$, such that:
\begin{equation}
M = \argmin_{\vec{p},r}{\sum_{i=1}^m e_{\vec{p}_i,r_i}},\quad I=\bigcup_{i=1}^m \Di{p}{r}{i}^I 
\label{eq:minimization}
\end{equation}
The disk regions can 



\subsection{AMAT as a Geometric Set Cover Problem}

\subsection{Extending AMAT to Textured Images}
\begin{itemize}
\item $L_0$ Smoothing
\item GANs (if that works)
\end{itemize}

\subsection{Grouping Medial Points to Branches}\label{sec:grouping}

% ==========================================================
\section{Experiments}\label{sec:experiments}
% ==========================================================
\subsection{Image Reconstruction (BSDS500)}
Experiments on image reconstruction using standard metrics (MSE, PSNR, and metrics using CNN feature similarity.

\subsection{Medial Point Detection (BMAX500)}\label{sec:medial-point-detection}
Recover all medial axes of the BSDS500. Compare with other skeletonization algorithms, such
as the CNN based method, and my previous MIL-based method

\subsection{Boundary Detection (BSDS500)}\label{sec:boundary-detection}
Medial axes to boundaries. 
MIL: compute an approximate scale based on the scale with maximum probability and try to recover boundaries.
CNN: same as above, this was already done in that paper for estimating segmentation masks using scale information.
We can compare the boundary recovery results as well.
EDGE DETECTORS: Standard edge detection algorithms (Dollar, HED, Canny, gPB etc: compare with algorithms specifically
tailored for this task. WE ARE NOT EXPECTING SOTA PERFORMANCE.

\subsection{Object Proposals}\label{sec:object-proposals}

\subsection{Implementation Details}

% ==========================================================
\section{Discussion}\label{sec:discussion}
% ==========================================================

{\small
\bibliographystyle{ieee}
\bibliography{iccv_2017}
}

\end{document}
